% $Header: /Users/joseph/Library/texmf/tex/latex/beamer/solutions/conference-talks/conference-ornate-20min.en.tex,v 90e850259b8b 2007/01/28 20:48:30 tantau $

\documentclass{beamer}
\graphicspath{{_plots/}}

\usepackage{animate}
\usepackage[makeroom]{cancel}
\usepackage{epstopdf}
\providecommand{\abs}[1]{\lvert#1\rvert}
\providecommand{\norm}[1]{\lVert#1\rVert}
%\newcommand{\rhoeq}{ \rho_{\mbox{eq}} }
\newcommand{\rhoeq}{ \rho_{\mbox{\scriptsize eq}} }
% This file is a solution template for:

% - Talk at a conference/colloquium.
% - Talk length is about 20min.
% - Style is ornate.



% Copyright 2004 by Till Tantau <tantau@users.sourceforge.net>.
%
% In principle, this file can be redistributed and/or modified under
% the terms of the GNU Public License, version 2.
%
% However, this file is supposed to be a template to be modified
% for your own needs. For this reason, if you use this file as a
% template and not specifically distribute it as part of a another
% package/program, I grant the extra permission to freely copy and
% modify this file as you see fit and even to delete this copyright
% notice.


\mode<presentation>
{
  \usetheme{Frankfurt}
  % or ...

  \setbeamercovered{transparent}
  % or whatever (possibly just delete it)
}


\usepackage[english]{babel}
% or whatever

\usepackage[latin1]{inputenc}
% or whatever

\usepackage{times}
\usepackage[T1]{fontenc}
% Or whatever. Note that the encoding and the font should match. If T1
% does not look nice, try deleting the line with the fontenc.


\title%[Short Paper Title] % (optional, use only with long paper titles)
{The f-wave method }

%\subtitle
%{Include Only If Paper Has a Subtitle}

\author%[Author, Another] % (optional, use only with lots of authors)
{Xin Yang, Hai Zhu}
% - Give the names in the same order as the appear in the paper.
% - Use the \inst{?} command only if the authors have different
%   affiliation.

\institute[University of Washington] % (optional, but mostly needed)
{
%  \inst{1}%
  Course project for Amath 574\\
  Department of Applied Mathematics\\
  University of Washington
}
%  \and
%  \inst{2}%
%  Department of Theoretical Philosophy\\
%  University of Elsewhere}
% - Use the \inst command only if there are several affiliations.
% - Keep it simple, no one is interested in your street address.

\date[03/12/2015] % (optional, should be abbreviation of conference name)
{Mar 12 2015}
% - Either use conference name or its abbreviation.
% - Not really informative to the audience, more for people (including
%   yourself) who are reading the slides online

%\subject{Applied Mathematics}
% This is only inserted into the PDF information catalog. Can be left
% out.

% If you have a file called "university-logo-filename.xxx", where xxx
% is a graphic format that can be processed by latex or pdflatex,
% resp., then you can add a logo as follows:

% \pgfdeclareimage[height=0.5cm]{university-logo}{university-logo-filename}
% \logo{\pgfuseimage{university-logo}}



% Delete this, if you do not want the table of contents to pop up at
% the beginning of each subsection:
%\AtBeginSubsection[]
%{
%  \begin{frame}<beamer>{Outline}
%    \tableofcontents[currentsection,currentsubsection]
%  \end{frame}
%}


% If you wish to uncover everything in a step-wise fashion, uncomment
% the following command:

%\beamerdefaultoverlayspecification{<+->}

\begin{document}

\begin{frame}
  \titlepage
\end{frame}

\begin{frame}{Outline}
  \tableofcontents
  % You might wish to add the option [pausesections]
\end{frame}


% Structuring a talk is a difficult task and the following structure
% may not be suitable. Here are some rules that apply for this
% solution:

% - Exactly two or three sections (other than the summary).
% - At *most* three subsections per section.
% - Talk about 30s to 2min per frame. So there should be between about
%   15 and 30 frames, all told.

% - A conference audience is likely to know very little of what you
%   are going to talk about. So *simplify*!
% - In a 20min talk, getting the main ideas across is hard
%   enough. Leave out details, even if it means being less precise than
%   you think necessary.
% - If you omit details that are vital to the proof/implementation,
%   just say so once. Everybody will be happy with that.

\section{Introduction}
\begin{frame}{What is the f-wave method?}
Idea is simple.
Wave-propagation form \\
\[
Q^{n+1}=Q^n-\frac{\Delta t}{\Delta x}(\sum s^p W^p)
\]
we call the wave $W$ w-waves. If instead we decompose $f$, we get f-waves.
\end{frame}

\begin{frame}{Implementation}
1.2..3...
\end{frame}

\begin{frame}{Why f-waves?}
Approximate Riemann solver. Conservation. Roe condition
\end{frame}

\section{Model setup}
\begin{frame}{1D Elastic wave equations}
\begin{align}
\epsilon(x,t)_t-u(x,t)_x=0 \label{strain}\\
\rho(x) u(x,t)_t-\sigma(\epsilon,x)_x=0 \label{n2}
\end{align}
first equation is definition of strain while the second one is the the newton's second law.
\end{frame}

\begin{frame}{Structure of solution to the Riemann problem}
Jacobian, eigenvalues and eigenvectors
physical meaning of the variables, maybe the effect of continuous  impedance and discontinuous impedance
\end{frame}

\begin{frame}{Approximate Riemann solver}
\end{frame}

\section{Periodic layerd media}
\begin{frame}
\begin{figure}
  \centering
  % Requires \usepackage{graphicx}
  \includegraphics[width=\textwidth]{frame41}\\
  \caption{stegoton}\label{ste}
\end{figure}
\end{frame}

\begin{frame}
% \href{run:travellingstegotons.gif}{Stegotons} 
 \href{run:travellingstegotons.gif}{Stegotons}
\end{frame}


% All of the following is optional and typically not needed.
\appendix
\section<presentation>*{\appendixname}
\subsection<presentation>*{For Further Reading}

\begin{frame}[allowframebreaks]
  \frametitle<presentation>{References}
  \begin{thebibliography}{10}
%  \beamertemplatebookbibitems
  % Start with overview books.
  \bibitem{bale2002}
  D. S. Bale, R. J. LeVeque, S. Mitran, and J. A. Rossmanith, SIAM J. Sci. Comput 24 (2002), 955-978.
  \bibitem{leveque2002}
  Finite Volume Methods for Nonlinear Elasticity in Heterogeneous Media
  by R. J. LeVeque, Int. J. Numer. Meth. Fluids 40 (2002), pp. 93-104.
  \bibitem{leveque2003}
  Randall J. LeVeque and Darryl H. Yong, SIAM J. Appl. Math., 63 (2003), pp. 1539-1560.
  \bibitem{ketcheson2012}
  David I Ketcheson, Randall J. LeVeque Comm. Math. Sci. 10 (2012), pp. 859-874.
  \end{thebibliography}
\end{frame}
\end{document}


