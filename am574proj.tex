\documentclass{article}
\usepackage{fullpage}
%%%%%%%%%%%%%%%%%%%%%%%%%%%%%%%%%%%
\usepackage{graphicx}
\usepackage{epstopdf}
\usepackage{amsthm}
\usepackage{amsmath}
\usepackage{amssymb}
\usepackage{caption}
\usepackage{subcaption}
\usepackage[all]{xy}



\everymath{\displaystyle}

\newenvironment{solution}
  {\begin{proof}[Solution]}
  {\end{proof}}
\newtheorem{definition}{Definition}
\newtheorem{example}{Example}
\newtheorem{theorem}{Theorem}
\newtheorem{remark}{Remark}
\newtheorem{lemma}{Lemma}
\newtheorem{inequality}{Inequality}
\newtheorem{proposition}{Proposition}

\providecommand{\abs}[1]{\left|#1\right|}
\providecommand{\norm}[1]{\lVert#1\rVert}
%%%%%%%%%%%%%%%%%%%%%%%%%%%%%%%%%%%
\begin{document}
\title{AMATH574 Conservation Laws and Finite Volume Methods\\ Winter Quarter 2015\\ Project Proposal: F-wave method for nonlinear equations with spatially varying fluxes.}
\author{Instructor : Professor Randall Leveque\\ Student: Hai Zhu, Xin Yang (yangxin@uw.edu)\\ Due date: Wednesday, Feb. 18, 2015}
%\date
\maketitle

\section{Abstract}
We study the f-wave method for elastic waves in heterogeneous media. 
\section{Introduction and Overview}
Objective:\\
\begin{enumerate}
\item Implement the f-method for nonlinear elastic waves in heterogeneous media. Reproduce some of the figures in \cite{bale2002} \cite{leveque2003} and \cite{ketcheson2012}.
\item Have a more comprehensive discussion of the f-wave method and problem. Mostly on the verifications of the details of the method mentioned in the paper and textbook. Some aspects include:
    \begin{itemize}
    \item The disadvantages of using cell-edge flux functions in wave-propagation algorithm metioned in \cite[p. 957]{bale2002}
    \item Justification of the Riemann solver used in \cite[p. 967]{bale2002}
    \item The non-conservation when using w-wave in wave-propagation algorithm for 1) nonlinear autonomous systems with simple Riemann solver (HLL?), 2) non-autonomous systems with "Roe average" Rieman solver.
    \item The slight difference of using $w=z/s$ in the limiter. p. 335 in the textbook.
    \item The new line of discontinuities caused by the discontinuities of the coefficient. \cite[p. 960]{bale2002}
    \item The breakdown of f-wave method in the case of singular flux (delta distribution). \cite[p. 961]{bale2002}
    \end{itemize}
\item If possible, have some discussion on the dispersive properties of layered media i.e, solitons and shocks.
\end{enumerate}
The importance of the goals is decreasing in the order.
\section{Theoretical Background}


\section{Computational Results}


\section{Summary and Conclusions}



%\section{Appendix A: MATLAB functions used and brief implementation explanation}
%\section{Appendix B: MATLAB codes}
%\section{Appendix C(optional)} Any algebraically intense calculations.

\begin{thebibliography}{9}
\bibitem{bale2002}
D. S. Bale, R. J. LeVeque, S. Mitran, and J. A. Rossmanith, SIAM J. Sci. Comput 24 (2002), 955-978.
\bibitem{leveque2003}
Randall J. LeVeque and Darryl H. Yong, SIAM J. Appl. Math., 63 (2003), pp. 1539-1560.
\bibitem{ketcheson2012}
David I Ketcheson, Randall J. LeVeque Comm. Math. Sci. 10 (2012), pp. 859-874.
\end{thebibliography}

\end{document}
